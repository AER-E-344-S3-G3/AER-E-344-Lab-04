\chapter{Discussion}
\label{cp:discussion}

\section{Coefficient of Pressure Distribution}

\autoref{fig:c_p_distribution} shows all the coefficient of pressure distributions plotted over the theoretical solution for the pressure distribution over a cylinder (see \autoref{ch:graphs} for all the individual pressure distribution graphs). All the pressures distributions but the \qty{5}{\hertz} pressure distribution track the theoretical solution well in the range of \qtyrange{150}{225}{\degree}. The phase shift in the high pressure part of the pressure distribution is likely due to a slight rotation of the cylinder in the wind tunnel (see \autoref{fig: Circular Cylinder Airfoil in the Wind Tunnel}). Outside the range \qtyrange{150}{225}{\degree}, the pressure distribution tracks the theoretical solution more poorly, but this is due to a limitation in the type of pressure measurement being taken. We noted that the pressure distributions all level out at a \gls{C_P} of approximately \num{-1}, which is the center of the theoretical sinusoid.

If we exclude the \qtyrange{5}{15}{\hertz} pressure distributions as shown in \autoref{fig:c_p_distribution_exc}, the pressure distributions track the theoretical solution better and become more uniform. This seems to indicate a significant amount of turbulence and separation is occurring in the \qtyrange{5}{15}{\hertz} flows. At around \qty{20}{\hertz} and above, the flow becomes more uniform.

\section{Coefficient of Drag}

Examining \autoref{fig:c_d_cylinder_re}, we see that the highest drag coefficient magnitude occurs at the lowest Reynolds number, when the motor frequency is at \qty{5}{\hertz}. We assume this larger drag coefficient is due to viscosity effects on the surface of the cylinder which are more significant at lower speeds. We suspect the large jump in \gls{C_d} between \qtylist{15;20}{\hertz} is due to turbulence and separation effects.