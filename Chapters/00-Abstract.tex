\thispagestyle{plain} % Page style without header and footer
% \pdfbookmark[1]{Resumo}{resumo} % Add entry to PDF
% \chapter*{Resumo} % Chapter* to appear without numeration
% \blindtext

% \keywordspt{Keyword A, Keyword B, Keyword C.}

% \blankpage

\pdfbookmark[1]{Abstract}{abstract} % Add entry to PDF
\chapter*{Abstract} % Chapter* to appear without numeration

By measuring the pressure at pressure taps arranged radially around a circular cylinder, we quantified the aerodynamic characteristics of a cylinder for different flow speeds. In the range of \qtyrange{150}{225}{\degree}—which corresponds to the side of the cylinder facing the flow—the distribution of the coefficient of pressure, \gls{C_P}, closely tracks the potential flow theory prediction. Outside this range, the experimental data does not track the theoretical prediction well, and the \gls{C_P} distribution levels out to approximately \num{-1}. Additionally, for each flow speed we calculated the aerodynamic drag coefficient from the pressure measurements and plotted these values as a function of the Reynolds number. The highest coefficient of drag we observed occurred at the lowest flow speed, possibly due to turbulence or viscous effects, which are more significant at lower Reynolds numbers.
